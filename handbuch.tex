\title{Wikidefine}
\author{Ren\'e Br\"uckner, Christian Frommert}
\date{\today}

\documentclass[12pt]{article}
\usepackage[
	colorlinks=true,
	urlcolor=blue,
	linkcolor=green
]{hyperref}

\usepackage{listings}
\usepackage{color}

\definecolor{dkgreen}{rgb}{0,0.6,0}
\definecolor{gray}{rgb}{0.5,0.5,0.5}
\definecolor{mauve}{rgb}{0.58,0,0.82}

\lstset{frame=tb,
  language=Java,
  aboveskip=3mm,
  belowskip=3mm,
  showstringspaces=false,
  columns=flexible,
  basicstyle={\small\ttfamily},
  numbers=none,
  numberstyle=\tiny\color{gray},
  keywordstyle=\color{blue},
  commentstyle=\color{dkgreen},
  stringstyle=\color{mauve},
  breaklines=true,
  breakatwhitespace=true,
  tabsize=3
}

\begin{document}
\maketitle

Wikidefine extrahiert Informationen aus Wikipedia und bildet zu jedem Artikel eine kurze Definition bestehend aus einer gegebenen Anzahl von S\"atzen.

\section{Einrichtung}

\subsection{Voraussetzungen}
Wikidefine setzt eine installierte Java Runtime Environment (JRE) in der Version 8.0 voraus. Außerdem muss entweder \texttt{git} installiert sein, falls es gecloned werden sollte, oder es muss \"uber einen Webbrowser von der Github Repository-Website heruntergeladen werden. Dar\"uber hinaus wird maven vorausgesetzt. \\\\
F\"ur den File Dump Extractor ben\"otigt man einen Wikipedia Dump. Diesen erh\"alt man von \url{https://dumps.wikimedia.org} . Von den verschiedenen Varianten wird der Dump ben\"otigt, der alle Seiten (pages) enth\"alt, inklusive Inhalt. F\"ur die englische Variante ist das beispielsweise enwiki-20170220-pages-articles.xml.bz2 , f\"ur deutsch: dewiki-20170220-pages-articles.xml.bz2 .

\subsection{Installation}
Zuerst das Git Repository clonen:
\begin{lstlisting}
git clone https://github.com/tm16wiki/wikidefine.git
\end{lstlisting}

Dann mit maven die Abh\"angigkeiten herunterladen und Wikidefine kompilieren:
\begin{lstlisting}
cd wikidefine
mvn install
\end{lstlisting}

Anschlie\ss{}end Wikidefine in der Shell starten:
\begin{lstlisting}
java -jar target/WikiDefine-*.jar
\end{lstlisting}

oder mit der GUI:
\begin{lstlisting}
java -jar target/WikiDefine-*.jar -gui
\end{lstlisting}

\section{Benutzung}
Man kann Wikidefine in der Shell oder mithilfe der GUI verwenden.

\subsection{Shell-Benutzung}
\subsubsection{Konfiguration}
Wikidefine muss nach dem Start zuerst konfiguriert werden. Die erste Konfiguration muss zudem "default" genannt werden, damit diese beim n\"achsten Start automatisch geladen wird. Nun wird die Sprache festgelegt (aktuell de oder en). Danach muss der Pfad zum Wikipedia Dump (XML) angegeben werden. Nun werden noch die Datenbankinformationen ben\"otigt. Unterst\"utzt werden SQLite PostgreSQL und MySQL. Falls man einen Dateipfad zu einer nicht existierenden SQLite Datenbank angibt wird automatisch eine neue SQLite Datenbank in diesem Pfad angelegt, sofern man Schreibrechte besitzt.

\subsubsection{Befehle - File Dumper}
Um den File Dumper Subprozess zu starten gibt man '\texttt{fd}' ein. Nun kann man den File Dump Extractor weiter konfigurieren:
\\\\
\begin{tabular}{| p{2cm} | p{3.5cm} | l | p{3cm} | p{4cm} |}
\hline
Option & Beschreibung & Befehl & Standardwert & Beispiel \\
\hline
Set Threads & Anzahl der zu benutzenden Threads & st & 4 & st 4 \\
\hline
Maximale Definitionen & Maximale Anzahl von zu extrahierenden Defintionen & sm & Integer. MAX\_VALUE & sm 5000 \\
\hline
Dateipfad & Dateipfad zum Wikipedia Dump im XML Format & sp & in der Hauptkonfiguration festgelegter Dateipfad & sp /home/user/wikidefine/dewikidump.xml \\
\hline
Datenbank- pfad & Pfad zur Datenbank & - & Datenbankpfad, der in der Hauptkonfiguration festgelegt wurde & Hier nicht \"anderbar (bitte in der Hauptkonfiguration \"andern) \\
\hline
Datenbank- export & Spezifiziert, ob die Definitionen in der Datenbank gespeichert werden sollen & se & true	& se \\
\hline
Debugging & Prozess zeigt akzeptierte und abgelehnte Definitionen w\"ahrend der Laufzeit an & sv & true & sv \\
\hline
Statistik & Zeigt am Ende des Prozesses die Laufzeit, die Anzahl der vorgefilterten Definitionen, die Anzahl der akzeptierten und abgelehnten Definitionen an & ss & false & ss \\
\hline
\end{tabular}
\\\\
Nach erfolgter Konfiguration startet man den Prozess mit dem Befehl \texttt{run}.
\\
Zum Beenden des Subprozesses gibt man \texttt{exit} ein.

\subsubsection{Befehle - Web Definition}
Um den Web Definition Subprozess zu starten gibt man '\texttt{wd}' ein. Nun kann man mit dem Befehl \texttt{sl} die Sprache festlegen, zum Beispiel \texttt{sl de} oder \texttt{sl en}.
\\
Um eine Definition zu crawlen gibt man nun \texttt{d "Wikipedia Titel"} ein. Beispiel: \texttt{d "Festplattenlaufwerk"}. Die Titel sind gleichzusetzen mit der URL (hier im Beispiel: https://de.wikipedia.org/wiki/\textbf{Festplattenlaufwerk})
\\\\
Zum Beenden des Subprozesses gibt man \texttt{exit} ein.
\subsection{GUI}
Um Wikidefine in der GUI zu starten gibt man den Parameter \texttt{-gui} beim Start an:
\begin{lstlisting}
java -jar target/WikiDefine-*.jar
\end{lstlisting}
\subsubsection{GUI - Konfiguration}
Im ersten Eingabefeld wird der Pfad zum Wikipedia Dump im XML Format angegeben. Durch einen Klick auf Open \"offnet sich ein Dateiauswahldialog. Darunter kann man die maximale Anzahl von Definitionen angeben, die Statistik, Debug-Informationen und Datenbank-Export an- und ausschalten. Auf der rechten Seite kann man die Anzahl der zu benutzenden Threads angeben sowie die gew\"unschte Sprache. Nachdem man die Informationen zur Datenbank angegeben hat kann man mit dem Play-Button auf der rechten Seite den Extraktionsprozess starten. Am unteren Rand der GUI sieht man dann eine Statusbar, die Auskunft \"uber den Fortschritt des Prozesses gibt.
\bibliographystyle{abbrv}
\bibliography{main}

\end{document}
This is never printed